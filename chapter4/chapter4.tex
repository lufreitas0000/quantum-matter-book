% File: chapter4/chapter4.tex
\chapter{The Hodge Theorem and Duality}
\label{chap:hodge_duality}

With the machinery of differential forms and curvature in hand, we can now explore the deep structure of the space of forms itself. This chapter introduces the Hodge theorem, a powerful result that provides a canonical way to decompose any differential form into a "harmonic" piece that reflects the global topology of the underlying manifold. This theorem illuminates the concept of duality in physical theories, most famously in Maxwell's equations, and has surprising applications in understanding subtle ordering phenomena in frustrated magnetic systems.

% --- Input the sections of this chapter from their respective files ---
\input{chapter4/sec1_hodge_star_laplacians.tex}
\input{chapter4/sec2_hodge_decomposition.tex}
\input{chapter4/sec3_duality_maxwell.tex}
\input{chapter4/sec4_order_by_disorder.tex}
```latex
% File: chapter4/sec1_hodge_star_laplacians.tex
\section{The Hodge Star Operator and Laplacians}
\label{sec:hodge_star_laplacians}

% Content for Section 4.1 will go here.
```latex
% File: chapter4/sec2_hodge_decomposition.tex
\section{The Hodge Decomposition Theorem}
\label{sec:hodge_decomposition}

% Content for Section 4.2 will go here.
```latex
% File: chapter4/sec3_duality_maxwell.tex
\section{Duality in Maxwell's Equations}
\label{sec:duality_maxwell}

% Content for Section 4.3 will go here.
```latex
% File: chapter4/sec4_order_by_disorder.tex
\section{Order by Disorder and Harmonic Modes in Frustrated Magnets}
\label{sec:order_by_disorder}

% Content for Section 4.4 will go here.
