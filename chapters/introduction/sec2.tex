\section{Factorization Algebras: A Rigorous Language for Locality}

At the heart of any quantum field theory is the principle of locality. This principle is most powerfully expressed through the \textbf{Operator Product Expansion (OPE)}. The OPE is the statement that when we take the expectation value of a product of two local operator insertions, $\mathcal{O}_1(x)$ and $\mathcal{O}_2(y)$, at nearby points, it can be expanded in terms of single operator insertions. Formally, the "equal" sign `~` in the expression below means that the equality holds inside any correlation function, once we integrate against smooth test functions.
\begin{equation}
    \mathcal{O}_1(x) \mathcal{O}_2(y) \sim \sum_k C_{12}^k(x-y) \mathcal{O}_k(y)
    \label{eq:ope_intro}
\end{equation}
Here, the $\{\mathcal{O}_k\}$ form a complete basis of local operators in the theory. The functions $C_{12}^k(x-y)$ are the OPE coefficients, which are universal numbers (for a CFT) or functions that typically diverge as the separation $x-y \to 0$.

While most famously associated with Conformal Field Theory (CFT), the OPE is a general feature of all quantum field theories. In high-energy physics, the OPE is a statement about the operator algebra at very short distances (the ultraviolet, or UV, limit). It allows one to understand how interactions behave at high energies by replacing products of operators with a simpler, local description. Conversely, it also has implications for the long-distance behavior (the infrared, or IR, limit) of the theory.

In condensed matter and quantum information, the OPE is a powerful tool for characterizing a system. The set of local operators $\{\mathcal{O}_k\}$ (often called the "operator content") and their fusion rules, encoded in the coefficients $C_{12}^k$, define the theory. For example, in a critical system described by a CFT, the scaling dimensions of the operators and the OPE coefficients are universal numbers that can be measured in experiments or numerical simulations. 


In quantum information, the OPE structure governs the fine-grained details of quantum entanglement, such as the entanglement spectrum. The collection of all operators and the complete data of their OPEs is what we mean by the "space of observables" of the theory. It is this rich structure that a factorization algebra is designed to capture.

\subsection*{A Pseudo-Definition}
\textbf{Factorization algebras} provide the modern and mathematically rigorous framework that captures the full structure of the OPE and the principle of locality. Let's build an intuitive definition. A factorization algebra, $\mathcal{F}$, is a rule that assigns a space of observables, $\mathcal{F}(U)$, to every well-behaved region $U$ of spacetime. This assignment must satisfy a crucial "factorization" property. If we have a collection of small, disjoint regions $U_1, \dots, U_n$ all contained within a larger region $V$, then there is a product map:
\begin{equation}
    \mu: \mathcal{F}(U_1) \otimes \dots \otimes \mathcal{F}(U_n) \to \mathcal{F}(V)
\end{equation}
This map is the rigorous formulation of the OPE. It tells us how to take observables localized in the small regions $U_i$ and multiply them to get a new observable in the larger region $V$.

The framework then imposes a \textbf{compatibility condition} on these maps. This condition is essentially associativity. It ensures that if we multiply three observables, the final result doesn't depend on which two we multiply first. For instance, the result of combining an operator in $U_1$ with the product of operators from $U_2$ and $U_3$ must be the same as combining the product of operators from $U_1$ and $U_2$ with the operator from $U_3$. The factorization algebra axioms ensure all such products are consistent as the geometry of the regions changes.

This structure allows for the direct computation of physical quantities like correlation functions. A vacuum state $\langle \cdot \rangle$ is a functional that assigns a number to an observable. A correlation function $\langle \mathcal{O}_1(x_1) \dots \mathcal{O}_n(x_n) \rangle$ is computed by taking the operators $\mathcal{O}_i \in \mathcal{F}(U_i)$, multiplying them using the factorization product $\mu$ to get a single operator $\mathcal{O}_{\text{prod}} \in \mathcal{F}(V)$, and then evaluating the vacuum on this result: $\langle \mathcal{O}_{\text{prod}} \rangle$.

\subsection*{Local and Non-Local Observables}
For a simple lattice model, like the 2D Ising model, the factorization algebra has a very concrete meaning. The local observables are the spin operators $\sigma_i$ at each site $i$, and products thereof. For a region $U$ on the lattice, the space of observables $\mathcal{F}(U)$ is simply the algebra generated by all spin operators $\sigma_i$ for sites $i \in U$. The factorization product is just the standard matrix multiplication of these operators. This provides a clear, discrete starting point. In the continuum limit, as the lattice spacing goes to zero at the critical point, this lattice algebra flows to the rich factorization algebra of the Ising CFT.

The true power of the framework is revealed when we consider more complex situations. A \textbf{Wilson loop} is a classic example of a non-local observable in a gauge theory. It is defined by taking the trace of the path-ordered exponential of the gauge connection $A$ around a closed loop $\gamma$: $W(\gamma) = \text{Tr} \left( \mathcal{P} e^{i \oint_\gamma A} \right)$. Physically, it measures the Aharonov-Bohm phase acquired by a heavy charged particle transported around the loop.

How does this fit into our local framework? A Wilson loop is not an element of $\mathcal{F}(U)$ for some small ball $U$. Instead, it is associated with the 1-dimensional loop $\gamma$ itself. This requires an extension of the factorization algebra concept, where we assign observables not just to open sets, but to submanifolds of various dimensions. However, even the basic framework can detect non-local phenomena. A Wilson loop can be seen as creating a "defect." In 2D topological phases, the analogue of a Wilson loop creates a flux at a point. While the loop operator itself is non-local, from the perspective of another excitation (an anyon) moving around it, the flux acts as a \textbf{point-like defect}. The anyon acquires a phase, and this interaction reveals the non-local topological order of the system.

This dependence on topology is captured by the algebra $\mathcal{F}(U)$ itself. Consider, as a concrete example, a theory of massive 2D Dirac fermions:
\begin{itemize}
    \item On a simple region like a \textbf{disk}, the theory is gapped. There are no low-energy excitations. Any local observable one could create has an energy cost of at least the gap size. Therefore, the algebra of low-energy observables on a disk is trivial: $\mathcal{F}(\text{Disk}) \cong \mathbb{C}$, containing only the identity operator.
    \item On an \textbf{annulus} (a disk with a hole), the situation is dramatically different. The hole in the manifold allows for non-trivial topology. We can thread a magnetic flux through the hole. The Dirac operator, when coupled to this flux, is no longer trivial. As dictated by the index theorem, it can have protected zero-energy modes bound to the boundaries of the annulus (the inner and outer circles). These zero-modes are new operators. The algebra of observables $\mathcal{F}(\text{Annulus})$ is no longer trivial; it is a fermionic algebra generated by these boundary modes.
\end{itemize}
In this way, the factorization algebra $\mathcal{F}$ acts as a probe, detecting the topology of the underlying spacetime region through the structure of its observable algebra. This deep connection between the algebra of observables and the geometry of the space is a central theme of this book.


