\section{The First Index Theorem: Index = Winding Number}
\label{sec:first_index_theorem}

We have now arrived at a critical juncture.
In the first section, we found a topological invariant: the winding number of a continuous map from the circle to itself.
In the second, we found an analytical invariant: the Fredholm index of an operator acting on an infinite-dimensional space.
Both are integers, and both are stable under small, continuous deformations of the underlying system.
The natural question arises: are these two numbers related?

The answer is a resounding yes, and their equality is the simplest manifestation of one of the most profound results in modern mathematics, the Atiyah-Singer Index Theorem.
We can state this first version of the theorem using the Toeplitz operator.

Let us recall the setup.
We begin with the Hilbert space $L^2(S^1)$ of all square-integrable functions on the circle, whose standard orthonormal basis is the Fourier basis $\{e^{in\theta}\}_{n\in\mathbb{Z}}$.
A key step is to split this space into two parts.
The \textbf{Hardy space}, denoted $H^2$, is the subspace of $L^2(S^1)$ consisting of functions with only non-negative Fourier frequencies.
A function $f(\theta) = \sum_{n\in\mathbb{Z}} c_n e^{in\theta}$ is in $H^2$ if and only if $c_n = 0$ for all $n < 0$.
This decomposition is fundamental in physics, often corresponding to a split between positive and negative energy states (or particle and anti-particle states).
We then let $P$ be the orthogonal projection operator that maps any function in $L^2(S^1)$ to its non-negative frequency part in $H^2$.

By forcing our operator to map the space of "positive energy" states to itself, we introduce a crucial structure.
The operator is built from two steps: first, multiplication by a function $\phi$, and second, projection back into the original space $H^2$.
The multiplication step, $f \mapsto \phi \cdot f$, can take a function with only non-negative frequencies and produce one with negative frequencies.
The projection $P$ then discards these newly created negative-frequency components.
It is this act of "projecting away" part of the result that gives the operator a chance to have a non-trivial kernel or cokernel.
If we simply considered the multiplication operator on the full space $L^2(S^1)$, its index would always be zero.
The restriction to the Hardy space is the essential ingredient.

From our continuous, non-vanishing function $\phi$ on the circle (the \textbf{symbol}), we construct the Toeplitz operator $T_\phi: H^2 \to H^2$.
The Atiyah-Singer Index Theorem, in this specific context, makes a remarkable claim.

\begin{theorem}[Index Theorem for Toeplitz Operators]
Let $\phi$ be a continuous, non-vanishing function on the circle $S^1$. The associated Toeplitz operator $T_\phi$ is a Fredholm operator, and its analytical index is equal to the negative of the winding number of its symbol:
\begin{equation}
    \text{Index}(T_\phi) = -\text{wind}(\phi, 0)
    \label{eq:toeplitz_index_theorem}
\end{equation}
\end{theorem}

\begin{proof}[Sketch of Proof]
    A full proof is beyond the scope of this introduction, but we can provide a compelling argument by considering the case where the symbol is a single Fourier mode, $\phi(z) = z^k$ for some integer $k$, where $z=e^{i\theta}$. The winding number of this symbol is exactly $k$. The theorem predicts an index of $-k$.
    
    The action of the operator on a basis function $e^{in\theta}$ (with $n \ge 0$) is $T_{z^k}(e^{in\theta}) = P(z^k e^{in\theta}) = P(e^{i(n+k)\theta})$.
    
    \textbf{Case 1: $k > 0$}.
    The operator maps $e^{in\theta}$ to $e^{i(n+k)\theta}$. Since $n \ge 0$ and $k > 0$, the result $n+k$ is always non-negative, so the projection $P$ is trivial.
    \begin{itemize}
        \item \textbf{Kernel:} $T_{z^k}$ maps the basis $\{e^{i0\theta}, e^{i1\theta}, \dots\}$ to $\{e^{ik\theta}, e^{i(k+1)\theta}, \dots\}$. No non-zero function is mapped to zero. The kernel is trivial, $\dim(\ker T_{z^k})=0$.
        \item \textbf{Cokernel:} The image of the operator is the subspace spanned by $\{e^{im\theta}\}_{m\ge k}$. The basis functions $\{e^{i0\theta}, e^{i1\theta}, \dots, e^{i(k-1)\theta}\}$ are missing from the image. There are $k$ such functions. The cokernel is therefore $k$-dimensional, $\dim(\text{coker} T_{z^k})=k$.
    \end{itemize}
    The index is $\text{Index}(T_{z^k}) = 0 - k = -k$. This matches the formula.
    
    \textbf{Case 2: $k < 0$}. Let $k = -j$ where $j > 0$.
    The operator maps $e^{in\theta}$ to $P(e^{i(n-j)\theta})$.
    \begin{itemize}
        \item \textbf{Kernel:} The projection gives zero if the frequency is negative, i.e., $n-j < 0$. Since $n \ge 0$, this is true for $n = 0, 1, \dots, j-1$. There are $j$ such basis functions. The kernel is $j$-dimensional, $\dim(\ker T_{z^{-j}})=j$.
        \item \textbf{Cokernel:} The operator maps the basis functions with $n \ge j$ to $e^{i(n-j)\theta}$, which generates all non-negative frequencies. The operator is surjective, so its image is all of $H^2$. The cokernel is trivial, $\dim(\text{coker} T_{z^{-j}})=0$.
    \end{itemize}
    The index is $\text{Index}(T_{z^{-j}}) = j - 0 = j = -k$. This again matches the formula.
    The general proof relies on showing that the index is stable under deformations of the symbol $\phi$, and any symbol can be deformed into the simple form $z^k$ where $k$ is its winding number.
\end{proof}

This theorem is a bridge between two worlds.
The left-hand side, the analytical index, is a property of an operator.
To compute it, one must solve for the dimensions of the kernel and cokernel—a task of analysis.
The right-hand side, the winding number, is a property of a path in the complex plane—a task of topology.
The theorem guarantees that the "hard" analytical calculation must give the same integer as the "easy" topological one.
The physical importance of this cannot be overstated.
It means that a physical quantity, like the number of zero-energy states, can be determined purely by the topological properties of the system's parameters, making it robust against small perturbations.

This result is the first step on a long and fruitful journey.
We will spend much of this book exploring its vast generalizations and physical consequences.
For now, we will turn to its first direct application in condensed matter physics: a simple one-dimensional model of a crystal that realizes this deep mathematical structure in a strikingly physical way.

% \subsection*{Exercises}
% \begin{exercise}[Index of a Dirac Operator]
%     A simple model for a relativistic particle in one dimension is the Dirac operator on the circle, $D = -i\sigma_2 \frac{d}{d\theta} + m\sigma_1$, acting on two-component spinor functions $\psi(\theta) = (\psi_1, \psi_2)^T$. Here, $\sigma_i$ are the Pauli matrices. The operator acts on the full Hilbert space $L^2(S^1) \otimes \mathbb{C}^2$.
    
%     The free operator ($m=0$) has eigenvalues $\lambda_n = n$ for each integer $n \in \mathbb{Z}$, with both positive and negative values. This suggests a natural split of the Hilbert space into $H_{>0}$ (positive energy states) and $H_{<0}$ (negative energy states).
    
%     Explain how the mass term $m\sigma_1$ can be viewed as a symbol, and how the index of an operator constructed from $D$ relates to the physical picture of energy levels crossing zero.
% \end{exercise}

% \begin{solution}
%     Let's analyze the operator in the Fourier basis (momentum space). The derivative becomes multiplication by the momentum $k$, where $k$ is an integer. The operator becomes a matrix for each $k$: $H(k) = k\sigma_2 + m\sigma_1 = \begin{pmatrix} 0 & -ik+m \\ ik+m & 0 \end{pmatrix}$.
%     The eigenvalues of this matrix are $\lambda_\pm(k) = \pm\sqrt{k^2+m^2}$.
    
%     The structure mentioned in the text becomes clear here. The Hilbert space is split by the free Dirac operator $H_0(k) = k\sigma_2$. Its eigenvalues are $\pm k$, so we can define the "positive energy" space as the one spanned by all eigenvectors with eigenvalue $k>0$, and the "negative energy" space (the "Dirac sea") by those with $k<0$. The $k=0$ states are special.
    
%     The full operator $D$ can be thought of as an operator that maps the positive-energy subspace to the negative-energy subspace. More formally, we can construct a Fredholm operator whose index counts the net number of states that cross zero energy as we vary a parameter.
    
%     The symbol of our operator can be taken as the off-diagonal component of the matrix $H(k)$, which is $\phi(k) = ik+m$. As we treat the momentum $k$ as a continuous variable that parameterizes the circle (the Brillouin zone), the function $\phi(k)$ traces a path in the complex plane.
    
%     If $m>0$, the path is the vertical line $m+ik$. This path never encircles the origin. Its winding number is 0. Correspondingly, the energy eigenvalues $\pm\sqrt{k^2+m^2}$ are never zero. There are no zero-energy states, and the index is 0.
    
%     If we were to vary the mass $m$, driving it from positive to negative, the path $\phi(k)$ would pass through the origin at the instant $m=0$ for $k=0$. As $m$ becomes negative, the path is now on the left side of the imaginary axis and still does not encircle the origin. The winding number remains 0.
    
%     This simple Dirac operator does not have a non-trivial winding number. However, the structure is the key. In the next section, we will see a model (the SSH model) whose Hamiltonian in momentum space is of the form $H(k) = d_x(k)\sigma_1 + d_y(k)\sigma_2$. The symbol is the complex function $\phi(k) = d_x(k) + id_y(k)$. If the path traced by this symbol as $k$ goes around the circle encloses the origin, its winding number will be non-zero. The index theorem will then guarantee the existence of zero-energy states, which physically manifest as robust states localized at the edges of the material.
% \end{solution}
