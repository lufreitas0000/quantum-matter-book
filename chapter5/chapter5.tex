% File: chapter5/chapter5.tex
\chapter{Clifford Algebras, Spinors, and the Dirac Equation}
\label{chap:clifford_spinors}

We now begin the second part of our journey, moving into the realm of relativistic quantum mechanics and the geometric structures required to describe fermions. The central new object is the spinor, a type of vector that behaves in a unique way under rotations. To construct spinors, we must first introduce Clifford algebras, which provide an elegant algebraic way to take the "square root" of the Laplacian operator. This naturally leads to the celebrated Dirac equation, the relativistic wave equation for the electron, which predicted the existence of antimatter and remains a cornerstone of modern physics.

% --- Input the sections of this chapter from their respective files ---
\input{chapter5/sec1_clifford_algebras.tex}
\input{chapter5/sec2_spin_groups_spinors.tex}
\input{chapter5/sec3_relativistic_dirac.tex}
\input{chapter5/sec4_majorana_fermions.tex}
```latex
% File: chapter5/sec1_clifford_algebras.tex
\section{Clifford Algebras and the ``Square Root'' of the Laplacian}
\label{sec:clifford_algebras}

% Content for Section 5.1 will go here.
```latex
% File: chapter5/sec2_spin_groups_spinors.tex
\section{Spin Groups and Their Representations (Spinors)}
\label{sec:spin_groups_spinors}

% Content for Section 5.2 will go here.
```latex
% File: chapter5/sec3_relativistic_dirac.tex
\section{The Relativistic Dirac Equation}
\label{sec:relativistic_dirac}

% Content for Section 5.3 will go here.
```latex
% File: chapter5/sec4_majorana_fermions.tex
\section{Majorana Fermions in Bogoliubov-de Gennes Systems}
\label{sec:majorana_fermions}

% Content for Section 5.4 will go here.
