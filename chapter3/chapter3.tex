% File: chapter3/chapter3.tex
\chapter{Connections, Curvature, and Gauge Theory}
\label{chap:connections_curvature}

Having established the basic language of manifolds and differential forms, we now introduce the central concepts of modern gauge theory: connections and curvature. These are the mathematical tools that describe how to compare vectors and fields at different points in a space, and they provide the universal language for describing forces in physics. We will see how electromagnetism is elegantly described as a U(1) gauge theory and how these same geometric structures can emerge in the low-energy description of complex condensed matter systems.

% --- Input the sections of this chapter from their respective files ---
% File: chapter3/sec1_principal_vector_bundles.tex
\section{Principal and Vector Bundles}
\label{sec:principal_vector_bundles}
\input{chapter3/sec2_connection_curvature.tex}
\input{chapter3/sec3_electromagnetism_u1.tex}
\input{chapter3/sec4_emergent_gauge_fields.tex}
```latex


% Content for Section 3.1 will go here.
```latex
% File: chapter3/sec2_connection_curvature.tex
\section{The Connection 1-Form and Curvature 2-Form}
\label{sec:connection_curvature}

% Content for Section 3.2 will go here.
```latex
% File: chapter3/sec3_electromagnetism_u1.tex
\section{Electromagnetism as a U(1) Gauge Theory}
\label{sec:electromagnetism_u1}

% Content for Section 3.3 will go here.
```latex
% File: chapter3/sec4_emergent_gauge_fields.tex
\section{Emergent Gauge Fields in Quantum Spin Ice}
\label{sec:emergent_gauge_fields}

% Content for Section 3.4 will go here.
