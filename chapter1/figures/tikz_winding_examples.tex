\begin{subfigure}[b]{0.3\textwidth}
    \centering
    \begin{tikzpicture}[scale=0.8]
        % The "pole" or origin
        \fill[black] (0,0) circle (2pt) node[anchor=south west] {$O$};
        % The path not enclosing the origin
        \draw[->, thick, blue, decoration={markings, mark=at position 0.5 with {\arrow{>}}}, postaction={decorate}]
            plot[smooth cycle, tension=0.8] coordinates {(1.5,0) (2,1.5) (1.5,2.5) (1,1.5)};
    \end{tikzpicture}
    \caption{$n=0$}
    \label{fig:n0}
\end{subfigure}
\hfill
\begin{subfigure}[b]{0.3\textwidth}
    \centering
    \begin{tikzpicture}[scale=0.8]
        % The "pole" or origin
        \fill[black] (0,0) circle (2pt) node[anchor=south west] {$O$};
        % The path enclosing the origin once
        \draw[->, thick, red, decoration={markings, mark=at position 0.25 with {\arrow{>}}}, 
        postaction={decorate}]
            (0,0) circle (1.5cm);
    \end{tikzpicture}
    \caption{$n=1$}
    \label{fig:n1}
\end{subfigure}
\hfill
\begin{subfigure}[b]{0.3\textwidth}
    \centering
    \begin{tikzpicture}[scale=0.8]
        % The "pole" or origin
        \fill[black] (0,0) circle (2pt) node[anchor=south west] {$O$};
        % The path enclosing the origin twice
        \draw[->, thick, green!60!black, decoration={markings, mark=at position 0.125 with {\arrow{>}}, mark=at position 0.625 with {\arrow{>}}}, postaction={decorate}]
           plot[smooth, tension=0.9, samples=20, domain=0:360] ({\x}: {1 + 0.5*cos(2*\x)});
    \end{tikzpicture}
    \caption{$n=2$}
    \label{fig:n2}
\end{subfigure}
