% \begin{tikzpicture}[
%     scale=1.2,
%     site/.style={circle, fill=black!20, minimum size=3mm, inner sep=0pt},
%     wannier/.style={ellipse, fill=red!30, opacity=0.6, minimum width=1.5cm, minimum height=0.8cm}
% ]
%     % Draw the lattice sites
%     \foreach \i in {0,...,6} {
%         \node[site] (S\i) at (\i*1.5, 0) {};
%     }

%     % Draw the potential and Wannier functions at different times
    
%     % t=0
%     \node at (3, 1.8) {$\phi=0$};
%     \draw[blue, thick] plot[smooth, domain=0:6.5] (\x, {0.5*cos(deg(2*\x*pi)) + 0.5});
%     \foreach \i in {0,1,2,3} {
%         \node[wannier] at (\i*3, 0.5) {};
%     }
    
%     % Arrow
%     \draw[->, very thick] (7, 0.5) -- (8, 0.5) node[midway, above] {Time};

%     % t=T/2
%     \node at (10.5, 1.8) {$\phi=\pi$};
%     \begin{scope}[xshift=9cm]
%         \foreach \i in {0,...,6} {
%             \node[site] (S\i) at (\i*1.5, 0) {};
%         }
%         \draw[blue, thick] plot[smooth, domain=0:6.5] (\x, {0.5*cos(deg(2*\x*pi - pi)) + 0.5});
%         \foreach \i in {0,1,2,3} {
%             \node[wannier] at (\i*3+1.5, 0.5) {};
%         }
%     \end{scope}

%     % Arrow
%     \draw[->, very thick] (3, -1) -- (3, -2) node[midway, right] {Cycle};

%     % t=T
%     \node at (3, -4.2) {$\phi=2\pi$};
%     \begin{scope}[yshift=-3.5cm]
%         \foreach \i in {0,...,6} {
%             \node[site] (S\i) at (\i*1.5, 0) {};
%         }
%         \draw[blue, thick] plot[smooth, domain=0:6.5] (\x, {0.5*cos(deg(2*\x*pi - 2*pi)) + 0.5});
%         % Show the charge has been pumped by one unit cell
%         \foreach \i in {1,2,3,4} {
%             \node[wannier] at (\i*3, 0.5) {};
%         }
%     \end{scope}

% \end{tikzpicture}

\begin{tikzpicture}[
    node distance=3cm % Vertical distance between figures
]

%----------------------------------------------------------------
% STYLES
%----------------------------------------------------------------
\tikzset{
    site/.style={
        circle, fill=black!70, minimum size=4mm,
        inner sep=0pt, draw=black, thick
    },
    % Wannier style with custom color and new width
    wannier/.style={
        fill=lightPink, % Use custom light pink color
        opacity=0.6,
        minimum width=1.8cm, % More localized as requested
        minimum height=0.7cm
    },
    % Potential style with a darker version of the custom blue
    potential/.style={
        color=lightBlue!85!black, % Darker blue for visibility
        %very thick,
        line width=2.0pt,
        smooth
    },
    % Arrow style with shorter tip
    flow-arrow/.style={
        -{Stealth[length=3mm, width=2.5mm]}, % Shorter arrow tip
        ultra thick,
        draw=black!80,
        shorten >=4mm,
        shorten <=4mm
    },
    state-label/.style={
        font=\Large,
        node distance=1cm
    }
}

%----------------------------------------------------------------
% MODULAR PIC
%----------------------------------------------------------------
\tikzset{
  pics/pump state/.style 2 args={
    code={
      \def\phase{#1}
      \def\chargeoffset{#2}
      \foreach \i in {0,...,5} {
        \node[site] (site-\i) at ({1.5*\i}, 0) {};
      }
      \draw[potential, samples=100, domain=-0.5:8]
          plot (\x, {0.6 * cos(deg( (2*pi/3)*\x + \phase )) + 0.3});
      \foreach \i in {0,1} {
        \node[wannier, shape=ellipse] at ({3*\i + \chargeoffset}, 0) {};
      }
    }
  }
}

%----------------------------------------------------------------
% DRAW THE FIGURE (VERTICAL ALIGNMENT)
%----------------------------------------------------------------
% 1. Create anchor nodes for the layout.
\node (state1) at (0,0) {};
\node (state2) [below=of state1] {};
\node (state3) [below=of state2] {};

% 2. Place the `pic`s at the anchor nodes.
\pic at (state1) {pump state={0}{0}};
\pic at (state2) {pump state={pi}{1.5}};
\pic at (state3) {pump state={2*pi}{3.0}};

% 3. Add labels and arrows relative to the anchors.
\node[state-label, left=of state1] {$\phi=0$};
\node[state-label, left=of state2] {$\phi=\pi$};
\node[state-label, left=of state3] {$\phi=2\pi$};

\draw[flow-arrow] (state1) -- +(0,-2.5);
\draw[flow-arrow] (state2) -- +(0,-2.5);

\end{tikzpicture}
