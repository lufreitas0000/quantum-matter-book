\section{The Winding Number as a Topological Invariant}
\label{sec:winding_number}

Let us begin with a simple question.
Consider a map from a circle to a circle.
We can visualize this by imagining a circular rope and a circular pole.
How many distinct ways can we wrap the rope around the pole?
Figure \ref{fig:winding_examples} illustrates this idea for several integer winding numbers.

\begin{figure}[htbp]
    \centering
    % Input the TikZ code from a separate file in the 'figures' subdirectory
    \begin{subfigure}[b]{0.3\textwidth}
    \centering
    \begin{tikzpicture}[scale=0.8]
        % The "pole" or origin
        \fill[black] (0,0) circle (2pt) node[anchor=south west] {$O$};
        % The path not enclosing the origin
        \draw[->, thick, blue, decoration={markings, mark=at position 0.5 with {\arrow{>}}}, postaction={decorate}]
            plot[smooth cycle, tension=0.8] coordinates {(1.5,0) (2,1.5) (1.5,2.5) (1,1.5)};
    \end{tikzpicture}
    \caption{$n=0$}
    \label{fig:n0}
\end{subfigure}
\hfill
\begin{subfigure}[b]{0.3\textwidth}
    \centering
    \begin{tikzpicture}[scale=0.8]
        % The "pole" or origin
        \fill[black] (0,0) circle (2pt) node[anchor=south west] {$O$};
        % The path enclosing the origin once
        \draw[->, thick, red, decoration={markings, mark=at position 0.25 with {\arrow{>}}}, 
        postaction={decorate}]
            (0,0) circle (1.5cm);
    \end{tikzpicture}
    \caption{$n=1$}
    \label{fig:n1}
\end{subfigure}
\hfill
\begin{subfigure}[b]{0.3\textwidth}
    \centering
    \begin{tikzpicture}[scale=0.8]
        % The "pole" or origin
        \fill[black] (0,0) circle (2pt) node[anchor=south west] {$O$};
        % The path enclosing the origin twice
        \draw[->, thick, green!60!black, decoration={markings, mark=at position 0.125 with {\arrow{>}}, mark=at position 0.625 with {\arrow{>}}}, postaction={decorate}]
           plot[smooth, tension=0.9, samples=20, domain=0:360] ({\x}: {1 + 0.5*cos(2*\x)});
    \end{tikzpicture}
    \caption{$n=2$}
    \label{fig:n2}
\end{subfigure}

    \caption{Three examples of paths in the plane, representing maps from a circle to a plane with a point removed at the origin $O$. The winding number $n$ counts the net number of times the path wraps counter-clockwise around the origin.}
    \label{fig:winding_examples}
\end{figure}

Our intuition suggests a straightforward answer.
We could not wrap it at all ($n=0$).
We could wrap it once ($n=1$), or twice ($n=2$), or three times.
We could also wrap it in the opposite direction, which we can denote by negative numbers.
It seems that for any integer $n \in \mathbb{Z}$, we can define a distinct way of wrapping the rope around the pole $n$ times.
The crucial insight is that we cannot smoothly transform a rope wrapped once into a rope wrapped twice without cutting it.
The number of times the rope is wrapped around the pole is a robust quantity; it is a topological invariant.
This integer is known as the \textbf{winding number}.

Let's make this mathematically precise.
A circle can be parameterized by an angle $\theta \in [0, 2\pi)$.
A map from a circle to a circle can therefore be represented by a function $f: S^1 \to S^1$, which we can write as a function $f(\theta)$ that is periodic, $f(\theta + 2\pi) = f(\theta)$.
Since the target space is also a circle, the value of the function, $f(\theta)$, can be thought of as a point on the unit circle in the complex plane, $e^{i\phi(\theta)}$.
Thus, our map is specified by a continuous, periodic function $\phi(\theta)$ such that $\phi(2\pi) = \phi(0) + 2\pi n$ for some integer $n$.

As we traverse the domain circle once (letting $\theta$ go from $0$ to $2\pi$), the function $f(\theta)$ traverses the target circle some integer number of times.
This integer is the winding number, and it can be extracted by the following integral formula:
\begin{equation}
    n = \frac{1}{2\pi i} \oint \frac{f'(\theta)}{f(\theta)} \, d\theta
    \label{eq:winding_number_integral}
\end{equation}
To understand this equation, let's represent our map as $f(\theta) = e^{i\phi(\theta)}$.
The derivative is then $f'(\theta) = i \phi'(\theta) e^{i\phi(\theta)}$.
Substituting this into the integral gives:
\begin{equation}
    n = \frac{1}{2\pi i} \oint \frac{i \phi'(\theta) e^{i\phi(\theta)}}{e^{i\phi(\theta)}} \, d\theta = \frac{1}{2\pi} \oint \phi'(\theta) \, d\theta = \frac{\phi(2\pi) - \phi(0)}{2\pi}
\end{equation}
This equation provides the story of the integral.
It tells us that the winding number $n$ is simply the total change in the phase $\phi(\theta)$ as we go around the circle once, normalized by $2\pi$.
Since the map must be continuous, the rope must meet up with itself after a full traversal of the pole.
This means that the total change in phase must be an integer multiple of $2\pi$, forcing $n$ to be an integer.

\subsection{The Winding Number as a 1-Form}
\label{subsec:winding_1_form}

There is a deeper, more geometric way to view the winding number that will be essential for the rest of this book.
Let's consider the target space of our map.
This is the unit circle, which lives in the plane $\mathbb{R}^2$.
The map itself, $f(\theta)$, traces out a path in this plane.
The winding number is non-zero only if this path encloses the origin.
This suggests that the origin is a special, singular point.
Let us consider the plane with the origin removed, a space we call the \textit{punctured plane}, $\mathbb{R}^2 \setminus \{0\}$.

The integrand in our formula, up to a factor of $i$, is $\frac{1}{2\pi} \phi'(\theta) d\theta$.
This is the pullback of a more fundamental object that lives on the punctured plane itself, the "angle 1-form".
A 1-form is simply an object that can be integrated along a path.
In Cartesian coordinates $(x,y)$, this 1-form is written as:
\begin{equation}
    \omega = \frac{1}{2\pi} \frac{-y\,dx + x\,dy}{x^2 + y^2}
\end{equation}
This expression may seem opaque, but it is precisely the object whose integral counts windings.
If we substitute $x = r\cos\theta$ and $y=r\sin\theta$, a direct calculation shows that $\omega = \frac{1}{2\pi}d\theta$.
The winding number of a closed path $\gamma$ is then given by the line integral:
\begin{equation}
    n = \oint_\gamma \omega
\end{equation}
This 1-form $\omega$ has a crucial property: it is \textit{closed}.
This means its exterior derivative is zero, $d\omega = 0$.
In the language of vector calculus, this is equivalent to saying that the corresponding vector field has zero curl.
A closed form has a special property: its integral between two points is locally independent of the path taken.
However, this is only true locally.
On the punctured plane, $\omega$ is famously not \textit{exact}.
An exact form is one that can be written as the total derivative of some global function, $\omega = dF$.
If $\omega$ were exact, then by Stokes' theorem, its integral around any closed loop would be zero.
But we know this is false; the integral of $\omega$ around the unit circle is 1.

This is the essence of topology entering the calculus.
The 1-form $\omega$ is closed but not exact on the punctured plane.
The failure to be exact is a direct consequence of the "hole" at the origin.
The winding number, as the integral of this form, is a topological probe that detects the presence of this hole.
This idea---that closed but not exact forms (called non-trivial cohomology classes) can detect the topology of a space---is the central idea of de Rham cohomology, a topic we will develop more formally in the next chapter.

\subsection*{Exercises}

\begin{exercise}
    In the complex plane, a point is described by $z = x+iy = re^{i\theta}$.
    Show that the angle 1-form $\omega = d\theta$ can be written in terms of $z$ and its conjugate $\bar{z}$.
    Use this to re-express the winding number integral as a contour integral in the complex plane.
\end{exercise}

\begin{solution}
    The key lies in the complex logarithm.
    For any complex number $z = re^{i\theta}$, we can write $\ln z = \ln r + i\theta$.
    The angle $\theta$ we seek is the imaginary part of the logarithm, $\theta = \text{Im}(\ln z)$.
    However, the logarithm is a multi-valued function.
    If we circle the origin, $\theta$ increases by $2\pi$, so $\ln z$ increases by $2\pi i$.
    To define a single-valued function, we must introduce a \textit{branch cut}, typically along the negative real axis, and define the principal value $\text{Arg}(z)$ which is restricted to the range $(-\pi, \pi]$.
    This function is discontinuous across the branch cut.
    Because of this discontinuity, there is no globally defined, single-valued function $F(z)$ on the punctured plane whose derivative is our angle 1-form.
    This is precisely why the 1-form $\omega = d\theta$ is not exact.
    However, the differential $d(\ln z)$ is a perfectly well-defined 1-form on the punctured plane.
    
    We start with $d(\ln z) = \frac{dz}{z} = \frac{dr}{r} + i\,d\theta$.
    Similarly, for the complex conjugate $\bar{z} = re^{-i\theta}$, we have $d(\ln \bar{z}) = \frac{d\bar{z}}{\bar{z}} = \frac{dr}{r} - i\,d\theta$.
    Subtracting the second equation from the first gives $\frac{dz}{z} - \frac{d\bar{z}}{\bar{z}} = 2i\,d\theta$.
    Solving for $d\theta$, we find the expression for the angle 1-form: $d\theta = \frac{1}{2i} \left( \frac{dz}{z} - \frac{d\bar{z}}{\bar{z}} \right)$.
    The winding number of a path $\gamma$ is given by $n = \frac{1}{2\pi} \oint_\gamma d\theta$.
    Substituting our new expression gives $n = \frac{1}{2\pi} \oint_\gamma \frac{1}{2i} \left( \frac{dz}{z} - \frac{d\bar{z}}{\bar{z}} \right)$.
    If the path $\gamma$ is on the unit circle, then $|z|^2 = z\bar{z} = 1$, which means $\bar{z} = 1/z$ and $d\bar{z} = -dz/z^2$.
    Substituting this gives $\frac{d\bar{z}}{\bar{z}} = z \left( -\frac{dz}{z^2} \right) = -\frac{dz}{z}$.
    The integral simplifies dramatically: $n = \frac{1}{2\pi} \oint_\gamma \frac{1}{2i} \left( \frac{dz}{z} - \left(-\frac{dz}{z}\right) \right) = \frac{1}{2\pi i} \oint_\gamma \frac{dz}{z}$.
    This is the famous Cauchy integral formula for the winding number, which by the residue theorem evaluates to an integer $n$ that counts how many times the path $\gamma$ winds around the pole at $z=0$.
\end{solution}
