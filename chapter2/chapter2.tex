\chapter{Manifolds, Cohomology, and the Aharonov-Bohm Effect}
\label{chap:manifolds_cohomology}

In the previous chapter, we saw how the winding number, a topological invariant, arose from studying maps on a circle. We caught a glimpse of a deeper structure when we rephrased it using the language of 1-forms on the punctured plane. This chapter formalizes that geometric language. We will develop the essential tools of differential geometry—smooth manifolds, differential forms, and de Rham cohomology—not as abstract mathematical exercises, but as the natural framework for describing physical phenomena like the Aharonov-Bohm effect and the behavior of electrons in a crystal. This will give us a much more powerful and general way to understand the origins of topological invariants in physics.

% --- Input the sections of this chapter from their respective files ---
\section{Smooth Manifolds and Differential Forms}
\label{sec:manifolds_forms}

% Content for Section 2.1 will go here.

\section{de Rham Cohomology}
\label{sec:deRham_cohomology}

% Content for Section 2.2 will go here.

\section{The Aharonov-Bohm Effect and Berry Phase}
\label{sec:ab_effect_berry_phase}

% Content for Section 2.3 will go here.

\section{Bloch's Theorem and Berry Curvature in the Brillouin Zone}
\label{sec:bloch_berry_brillouin}

% Content for Section 2.4 will go here.

