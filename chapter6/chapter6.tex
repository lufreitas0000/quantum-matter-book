% File: chapter6/chapter6.tex
\chapter{Spin Geometry and Gravity}
\label{chap:spin_geometry}

Having defined spinors on flat space, we now generalize these concepts to curved manifolds. This requires the introduction of a spin structure, a topological choice that must be made on a manifold before one can consistently define spinors. With this structure in place, we can define the Dirac operator on a curved background, which describes how fermions behave in the presence of gravity. This framework is not only essential for quantum field theory in curved spacetime but also provides a powerful language for classifying topological defects and textures in ordered phases of condensed matter.

% --- Input the sections of this chapter from their respective files ---
\input{chapter6/sec1_spin_structures.tex}
\input{chapter6/sec2_dirac_operator_curved.tex}
\input{chapter6/sec3_fermions_gr.tex}
\input{chapter6/sec4_classification_defects.tex}
```latex
% File: chapter6/sec1_spin_structures.tex
\section{Spin Structures on Manifolds}
\label{sec:spin_structures}

% Content for Section 6.1 will go here.
```latex
% File: chapter6/sec2_dirac_operator_curved.tex
\section{The Dirac Operator on a Curved Background}
\label{sec:dirac_operator_curved}

% Content for Section 6.2 will go here.
```latex
% File: chapter6/sec3_fermions_gr.tex
\section{Fermions in General Relativity}
\label{sec:fermions_gr}

% Content for Section 6.3 will go here.
```latex
% File: chapter6/sec4_classification_defects.tex
\section{Classification of Topological Defects and Textures in Ordered Media}
\label{sec:classification_defects}

% Content for Section 6.4 will go here.
