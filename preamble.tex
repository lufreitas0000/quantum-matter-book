% PREAMBLE FOR "Factorization Algebras, Geometry, and Categories for Quantum Matter"

%------------------------------------------------------------------
% Font and Encoding Setup (Requires XeLaTeX or LuaLaTeX)
%------------------------------------------------------------------
\usepackage{fontspec} % Allows loading of system fonts

% --- Define the Body and Display Fonts ---
% Assumes a 'fonts' folder in the project containing 'seguivt.ttf' and 'seguivd.ttf'.
\setmainfont{seguivt.ttf}[
    Path = ./fonts/,
    Ligatures = TeX
]
\newfontfamily\displayfont{seguivd.ttf}[
    Path = ./fonts/,
    Ligatures = TeX
]

%------------------------------------------------------------------
% Page Layout and Geometry
%------------------------------------------------------------------
\usepackage[a4paper, margin=1in]{geometry}
\usepackage{setspace}
\setstretch{1.4}

%------------------------------------------------------------------
% Header and Title Formatting
%------------------------------------------------------------------
\usepackage{titlesec}
\titleformat{\part}{\normalfont\displayfont\bfseries\Huge}{\thepart}{1em}{}
\titleformat{\chapter}[display]
  {\normalfont\displayfont\bfseries\huge}
  {\chaptertitlename\ \thechapter}
  {20pt}
  {\Huge}
\titleformat{\section}
  {\normalfont\displayfont\bfseries\Large}
  {\thesection}
  {1em}
  {}
\titleformat{\subsection}
  {\normalfont\displayfont\bfseries\large}
  {\thesubsection}
  {1em}
  {}
\titleformat{\subsubsection}
  {\normalfont\displayfont\bfseries}
  {\thesubsubsection}
  {1em}
  {}

%------------------------------------------------------------------
% Core Mathematics and Physics Packages
%------------------------------------------------------------------
\usepackage{amsmath}
\usepackage{amsthm}
\usepackage{mathtools}
\usepackage{unicode-math}
\setmathfont{Latin Modern Math}

\usepackage{physics}
\usepackage{slashed}
\usepackage{bm}

%------------------------------------------------------------------
% Graphics and Figures
%------------------------------------------------------------------
\usepackage{graphicx}
\usepackage{xcolor}
\usepackage{tikz}
\usetikzlibrary{
    arrows.meta,
    calc,
    positioning,
    decorations.markings,
    patterns,
    quantikz,
    shapes.geometric % *** ADDED THIS LIBRARY FOR ELLIPSE SHAPE ***
}
\definecolor{lightPink}{HTML}{F5A9B8}
\definecolor{lightBlue}{HTML}{5BCEFA}

\usepackage{caption}
\usepackage{subcaption}

%------------------------------------------------------------------
% Tables and Lists
%------------------------------------------------------------------
\usepackage{booktabs}
\usepackage{enumitem}

%------------------------------------------------------------------
% Referencing and Bibliography
%------------------------------------------------------------------
\usepackage[colorlinks=true, urlcolor=blue, linkcolor=black, citecolor=blue]{hyperref}
\usepackage{cleveref}
\usepackage[backend=biber, style=numeric-comp, sorting=none]{biblatex}
\addbibresource{references.bib}

%------------------------------------------------------------------
% General Formatting and Utilities
%------------------------------------------------------------------
\usepackage{microtype}

%------------------------------------------------------------------
% Custom Theorem-like Environments (Updated with new styling)
%------------------------------------------------------------------
% Style for definitions and examples
\theoremstyle{definition}
\newtheorem{definition}{Definition}[chapter]
\newtheorem{example}{Example}[chapter]

% A new style for exercises with a bold label
\newtheoremstyle{exercise_style}
  {\topsep}   % Space above
  {\topsep}   % Space below
  {\normalfont} % Body font
  {}          % Indent amount
  {\bfseries} % Theorem head font
  {.}         % Punctuation after theorem head
  {.5em}      % Space after theorem head
  {}          % Theorem head spec (can be left empty)
\theoremstyle{exercise_style}
\newtheorem{exercise}{Exercise}[section]

% Style for theorems, lemmas, etc.
\theoremstyle{plain}
\newtheorem{theorem}{Theorem}[chapter]
\newtheorem{lemma}[theorem]{Lemma}
\newtheorem{proposition}[theorem]{Proposition}
\newtheorem{corollary}[theorem]{Corollary}

% Style for remarks
\theoremstyle{remark}
\newtheorem{remark}{Remark}[chapter]

% Custom environment for solutions with smaller, single-spaced text
\newenvironment{solution}{%
  \par\addvspace{\topsep}%
  \noindent\textbf{Solution. }%
  \begingroup % Start a group to keep changes local
  \small
  \setstretch{1.0} % Set single spacing
}{%
  \qed\par\addvspace{\topsep}%
  \endgroup % End the group, restoring font size and spacing
}

% Redefine the proof environment to match the solution style
\renewenvironment{proof}[1][\proofname]{%
  \par\addvspace{\topsep}%
  \noindent\textbf{#1. }%
  \begingroup % Start a group to keep changes local
  \small
  \setstretch{1.0} % Set single spacing
}{%
  \qed\par\addvspace{\topsep}%
  \endgroup % End the group, restoring font size and spacing
}


%------------------------------------------------------------------
% Book Information
%------------------------------------------------------------------
\title{Factorization Algebras, Geometry, and Categories for Quantum Matter}
\author{An Expert Physicist and Master Educator}
\date{\today}
